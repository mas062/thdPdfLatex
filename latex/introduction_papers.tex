\chapter{Introduction to the papers}

\pagebreak
\section{Introduction}

The main scientific part of this thesis consists of three papers. They come in a sequence to show the research progress within this PhD program. Paper I includes a detailed study of how variations in geological parameters impact the evolution of the injected \coo\ plume. Knowing the migration path of the plume is essential if one wants to assess  the risk for \coo\ leaving the aquifer through imperfections in the caprock or through open lateral boundaries. Second, to determine the fasibility of the injection process and reduce the potential for introducing fracturing during the injection process, it is crucial to know the pressure buildup. Likewise, it is important to know how far pressure pulses induced during injection propagate beyond the zones invaded by the injected \coo. Therefore, a special study is dedicated to pressure analysis in the system. This is reported in Paper II, which is submitted to the International Journal of Greenhouse Gas Control (IJGGC). Finally, Paper III reports modern stochastic techniques used to perform detailed quantitative sensitivity analysis and probabilistic risk assessments. This paper is accepted for publication in the IJGGC. This paper was submitted for publication earlier than the second paper.

\section{Summary of papers}

\paperitem{I}
{Impact of geological heterogeneity on early-stage \coo\ plume migration: \coo\
spatial distribution sensitivity study}
{
\textbf{Summary:}

We use a set of SAIGUP realizations selected to cover the variability of five
sedimentological and structural geological parameters. The selected parameters are
lobosity, barriers, aggradation angle, progradation direction, and faults. Each
of these parameters varies over three levels, except the progradation direction,
which includes up-dip and down-dip directions. Combining the available 
parameters makes $162$ realizations. However, two cases were missing in the original setup. Therefor, $160$ geological realizations are used here.

$30~\mbox{years}$ of injection and $70~\mbox{years}$ of early migration of \coo\ are simulated and flow responses related to the storage
capacity and leakage risk objectives are defined and calculated from
the simulation results. The responses are reported in scatter plots at the end of
injection and at the end of early migration time. 

This work is specific in examining how heterogeneity influences flow behavior
by using a number of geological realizations. Flow responses defined in
this work are specific to \coo\ studies and differ from the responses
used in the original SAIGUP project to study oil recovery. We simulate the
aquifer average pressure, residual and mobile CO$_2$
saturation, and spatial distribution of connected CO$_2$ volumes. These
responses can be considered to evaluate the site storage capacity and risk of
CO$_2$ leakage to surface. 

The injector is controlled by a constant rate and no pressure constraint is set
to allow for all ranges of pressure, including those that are unrealistic.
Moreover, we define an additional model output that is related to the risk of
CO$_2$ leakage through any breakings in the cap-rock.

Finally, we perform a quantitative sensitivity analysis by using the flow simulation results. The sensitivity analysis results suggest that aggradation angle, fault criteria, and progradation direction are the most influential geological parameter in our study.

In this work, we clearly see the range of variations in the flow responses that demonstrates how important it is to model the geological features accurately. 

\vspace{0.5cm}
\noindent\textbf{Comments:}
%--1
This work initially was presented at the ACM conference in Edinburgh,  2010. More details of the work are reported in proceedings for the CMWR conference in Barcelona, 2010 and in the ECMOR conference in Oxford, 2010. The final version is submitted to the Goundwater. 

The following comments are important to be considered here:

\begin{itemize}

\item \textit{The SAIGUP realizations}
 
Topography is a major player in the gravity dominated flow behavior. The SAIGUP
realizations include variability in topography of the geological layering via
structural changes due to faults and also barriers in the model. These are good
enough for early migration when the \coo\ and water segregate and plumes
accumulate below cap-rock and start the longer migration. In the long-term
migration, top surface geometry is an important geological parameter and
larger models than the SAIGUP models with a better resolution of the top surface
are needed to get good predictions of the long-term migration phase. This
was considered in the next generation of geological studies performed following
this study \cite{syversveenstudy,nilsen2012impact} under the IGEMS research
project.

\item \textit{Physical assumptions}
 
The work concentrates on how geological heterogeneity impacts the flow
performance. We need to measure the volumetric sweep efficiency of $\mbox{CO}_2$
plumes to evaluate the residual trapping. Including more
physics in the modeling will add the computational costs specially when the flow
modeling is used in a sensitivity analysis or risk assessment process.
Therefore, we used simple fluid models for PVT. 

\item \textit{Uncertainty considerations}

Our main motivation for using the SAIGUP data was the extensive work that was put into building realistic geological realizations. The geological parameters are changed in value between low and high levels. These values are assumed with the same probability. In general, this probability might not be uniform, depending to the regional geology of the storage site. 

\end{itemize}
%--3
Within one geological realization, injection location can dramatically impact the injectivity of the well. In fact, this is an uncertain parameter in the \coo\ storage operations. Choosing to inject in the river channels or in the permeable homogeneous parts near the shore will enhance the injectivity and the \coo\ sweep efficiency in the medium. On the other hand, injecting in locations with low permeabilities and pore-volumes can significantly increase the injection pressure, while limiting the transport of \coo\ in the medium. Studying the impact of injection location can be performed by injecting in many different points in one realization and comparing the corresponding flow responses. However, this will considerably increase the number of detailed simulations in the study. 

For the allowed time, we limited our study to a fixed point by injecting via one well in the flank part of the SAIGUP models. This location is selected after qualitative analysis of a detailed study on a homogeneous case. There, we aimed to fulfill the criterion of maximizing the \coo\ storage capacity via increasing the vertical travel path toward the structural trap location under the cap-rock. One mitigating strategy for minimizing the effect of injection location can be to inject via several wells in different locations in the medium.  

Similar argument applies to the leakage risk study reported here. We use a leakage probability over the cap-rock that can dramatically influence the calculated leakage risk. We take this assumption to simplify the way we introduce the method. 

\vspace{0.5cm}
\noindent\textbf{Contribution of the candidate:}

The idea of using realizations from the SAIGUP project to study how variations in geological parameters impact the injection and early-stage migration of CO2 was first suggested by the main supervisor of this thesis. The conceptual design of the injection scenario, as well as the measured reservoir responses were developed jointly with the co-authors of the paper. The candidate was solely responsible for working out the details of the simulation setup, developing a work-flow, performing simulations, post-processing results, and developing the first analysis of the results. The candidate then collaborated with the co-authors to refine the analysis and write the paper.
}

\paperitem{II}
{Geological storage of CO$_2$: heterogeneity impact on pressure behavior}
{\textbf{Summary:}

Pressure build-up is an important criterion that can determine the success and failure of
$\mbox{CO}_2$ storage operations. Over-pressurized injections can induce
new fractures and open the existing faults and fractures that increases the risk
of leakage for the mobile \coo\ in the domain. On the other
hand, the pressure disturbance imposed on the system travels within the
domain beyond the scales of \coo\ distribution. If the CO$_2$ is
injected into a saline aquifer connected to fresh water aquifers, the pressure
pulse may result in fresh water contaminations by the brine far from the
injection point. We define specific pressure responses to examine the pressure
disturbance in the system during injection.

Two injection scenarios are examined for the same $160$ geological realizations
setup. In the first scenario, the injector is set to a fixed volumetric rate to
inject the \coo\ volume in $30$ years into the domain, allowing for an
unlimited pressure build-up. In the second scenario, a pressure constraint is
set on the injector that results in various rate of injection in different
geological realizations to inject the same amount of \coo\ volume
considered in the first injection scenario. 

Pressure response sensitivity study with respect to different geological
features indicates the significance of aggradation angle, progradation
direction, and faults during injection. A probabilistic pressure analysis is
also performed based on the $160$ simulations on the available realizations.

\vspace{0.5cm}
\noindent\textbf{Comments:}

The results reported in this paper can vary by choosing different boundary conditions for the model and different model size. We choose open boundaries on three sides of the model. In general, pressure values can be larger than those that are simulated here.

Well location is chosen to be fixed in our study. Choosing different location of injection in the model can result in a dramatically different pressure behavior. We use one injector in the study to simplify the pressure analysis. To investigate the effect of well location on the results, one can inject via many injectors. Other option is to study the impact of changing the well location in a single injector model.

Finally, the early pressure build up that happens around the well is due to the low \coo\ saturations existing near the injector in the beginning of injection.
 This build-up is sensitive to the grid resolution around the injector. The simulated pressures can be less if we use finer grid near the injector. In some experiments that is not reported in the paper, we concluded that, with the grid used in our study, this difference is not very dramatic.}

\paperitem{III}
{Geological storage of $\mbox{CO}_2$: global sensitivity analysis and risk
assessment using the arbitrary polynomial chaos expansion}
{\textbf{Summary:}

In this paper, we perform a stochastic sensitivity and risk analysis. We obtain
a high resolution global sensitivity and probabilistic study on the flow
responses that are defined and discussed in the previous papers. We choose
barriers, aggradation angle, and
faults from the SAIGUP geological parameters. Faults are considered by changing
the transmissibility value across them, which
is a continuous parameter. One more parameter is added to the study which is
common in the literature and models the external pressure support from other
aquifers attached to the model (regional groundwater effect).

Flow simulation on high resolution variational geology demands a huge
computational costs. To enhance the calculation speed, we use a data-driven
method that does not need to solve the full physical flow equations. We
approximate the flow solver by a response surface method that is a polynomial
and relates the system output to the input with a minimal computational cost.
We use the arbitrary polynomial chaos expansion (aPC) to approximate the
flow responses. The aPC method considers the uncertainty in the input variables.

A global sensitivity analysis is performed by employing Sobol indices that are
based on variances of responses. The method is shown to be robust in problems
of high levels of complexity and non-linearity.  

And finally, we perform a Monte-Carlo process using the approximating polynomial
on a high resolution input variations. This makes it possible to perform a high
resolution probabilistic study on the flow responses. This way, extreme cases
can be identified by probability of occurrence.

\vspace{0.5cm}
\noindent\textbf{Comments:}

This work was presented in the proceedings of the European Geosciences Union (EGU) General Assembly 2012, April, Vienna, Austria, Geophysical Research Abstracts., Vol. 14, EGU2012-9243. The detailed report is accepted for publication in the International Journal of
Greenhouse Gas Control (IJGGC), in May 2013, http://dx.doi.org/10.1016/j.ijggc.2013.03.023.
 
To implement our stochastic technique, we choose geological parameters
in this study that can be interpolated between two levels of their values. For
example, it makes sense to use barriers coverage level of $25\%$ between the low
($10\%$) and medium ($50\%$) levels used in the previous studies. Some of the
geological parameters are discrete and can not be interpolated between two
values. For instance, lobosity can only be varied over three points and we can
not define a $1.5$-lobe. 

Having a large number of points in the input values interval requires
intensive geological modelings to be used in the flow simulations. Using the
data-driven polynomial, the approach only needs evaluating the polynomial in the
defined values, and there is no need for full geological modeling except in
the collocation points, i.e., point values that the polynomial coefficients must
be calculated. 

The work reported here is to demonstrate the work-flow of using the aPC for
global sensitivity analysis and probabilistic risk assessment. A normal
work-flow starts by defining the uncertainties in the input parameters and
follows by building the geological models for the aPC collocation points that are based on those uncertainties.
To perform this study on the SAIGUP models that are consistent
with a uniform uncertainty in the geological parameters, with no loss of
generality, we used uniform uncertainty distributions for our study. However,
the aPC method is not limited to uniform uncertainty descriptions.

Geological features are ranked based on the sensitivity analysis results. The
results are in agreement with dynamics of the flow in the aquifer. Aggradation
angle is the most influential parameter, while the regional groundwater has the
least influence in the model responses. The study is not limited to the assumed
uncertainty of input parameters and the conclusion may change for a very
different uncertainty description. 

\vspace{0.5cm}
\noindent\textbf{Contribution of the candidate:}

The study was a joint work between the candidate and the co-authors on the following steps:

\begin{itemize}
\item Defining the problem.
\item Designing the simulation scenarios.
\item Designing the work-flow.
\item Integrating the aPC MATLAB code into the work-flow.
\item Performing the runs and processing the results.
\item Performing the global sensitivity analysis.
\item Performing the risk assessment.
\item Analyzing the results and preparing plots.
\item Writing the report.
\end{itemize}

The candidate had a solid and major contribution in every step, and in particular, integrating the aPC code into the work-flow, running the simulations, performing the sensitivity and risk analysis, and processing the results. The report has gone through extensive reviews. 
}
