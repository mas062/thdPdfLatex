\chapter{Abstract}

Geological \coo\ storage is a key technology to be utilized for preventing the industrial \coo\ emission into the atmosphere. A successful storage operation requires a safe geological structure with large storage capacity. The practicality of the technology is challenged by various operational concerns from site selection to monitoring the injected \coo\ in the long-term. 

The research in this report is aimed to address the significance of a sophisticated geological modeling that can help in prediction of the storage performance. In the first part, we investigate the significance of assessing the geological uncertainty and its consequences in the site selection and early stages of the storage operations. This includes the injection period and the early migration time of the injected \coo\ plume. The key part of the research is the extensive set of realistic geological realizations used in the analysis. The
heterogeneity is modeled by the most influential geological parameters in a shallow-marine system. Among those parameters are the aggradation angle, levels of barriers in the system, faults,
lobosity, and progradation direction. 

A typical injection scenario is simulated over $162$ realizations and major flow responses are defined to measure the success of the early stages of \coo\ storage operations. These responses include the volume of trapped \coo\ by capillarity, dynamics of the plume in the medium, pressure responses, and risk of leakage through a failure in the sealing cap-rock. Impact of geological uncertainty on these responses is investigated by comparing all cases for their performance. The results show a large variations in the responses due to changing the geological parameters. Among the main influential parameters are the aggradation angle, the progradation direction, and the faults in the medium.
 
A sophisticated geological uncertainty study requires large number of detailed simulations that are time consuming and computationally costly. The second part of the research introduces a work-flow that employs an approximating response surface method, which is called arbitrary polynomial chaos (aPC).  The aPC is fast and sophisticated enough to be used practically in the process of sensitivity analysis and uncertainty and risk assessment. We demonstrate the work-flow by combining the aPC with a global sensitivity analysis technique, namely the Sobol indices, which is a variance based method and has proven to be practical for sophisticated problems. The probabilistic uncertainty analysis is performed by applying the Monte-Carlo process using the aPC. The results show that the aPc can be used successfully in an extensive geological uncertainty study.
