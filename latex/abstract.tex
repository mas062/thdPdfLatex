\chapter{Abstract}

In the context of geological CO$_2$ storage, one main uncertainty that makes
the prediction of operational success more complicated, is the geological
heterogeneity in the system. The CO$_2$ storage capacity and risk of leakage
from the storage location are highly sensitive to the geological modeling of
the problem. The main motivation of our works is to address the
importance of proper geological modeling and to provide a practical work-flow
for assessing the geological uncertainty consequences in the operations.

We choose the shallow-marine depositional system for our studies, but the same
method could be implemented for any other types. The study is based on large
number of geological realizations with different levels of heterogeneity. The
heterogeneity is modeled by the main geological parameters and used to discuss
the flow responses that are important in the storage of CO$_2$. Among those
parameters are the aggradation angle, levels of barriers in the system, faults,
lobosity, and progradation direction. 

We describe the flow behavior in terms of flow responses that can be used to
evaluate the performance of geological storage of CO$_2$ in the aquifers and
abandoned oilfields. The injected plumes of CO$_2$ are analyzed for their
volumes and numbers and their dynamics in the system. Detailed sensitivity
analysis and risk assessment are performed on the designed injection and early
migration study. The geological parameters are ranked based on their influence
on the flow response variations. No general conclusion for uncertainty
assessment is expected from our studies, since this may change in different
geological regions. However, we demonstrate a practical work-flow that can be
used in any uncertainty assessment project.

We mainly consider the injection and early
migration of CO$_2$ as this work is part of a project that covers the long-term
CO$_2$ migrations. Nevertheless, the discussions and results here can be used
with some modifications (such as extending the model spatial extent) in the
long-term migrations.


The work is presented and published in many scientific conferences in the form
of proceedings, posters, and seminars. Some parts are submitted to the
literature and are in their way to be published in the literature.